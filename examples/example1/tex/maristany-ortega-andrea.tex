\documentclass[11pt,catalan]{article}

\usepackage[ca,NIC]{examen}

\data{11 de novembre de 2020}
\examen{Primer Parcial}
\quadrimestre{Tardor}
\curs{2020-2021}
\assignatura{\`Algebra Lineal}
\cognoms{Maristany Ortega}
\nom{Andrea}

\begin{document}

\begin{enunciat}
Trobeu la inversa de la matriu 
\[
  A = \begin{pmatrix}{*{3}r} -3 & 1 & 2\\ 0 & 1 & 0\\ 0 & 0 & -2\end{pmatrix}\,.
\]
\end{enunciat}

\begin{quadricula}
\begin{tabular}{|L{5cm}{1.25cm}|}
\hline
$A^{-1}=$ \\
\hline
\end{tabular}
\end{quadricula}

\begin{solucio}
\begin{center}
\begin{tabular}{|C{5cm}{1.25cm}|}
\hline
$A^{-1}=\deufrac{1}{6}\begin{pmatrix}{*{3}r} -2 & 2 & -2\\ 0 & 6 & 0\\ 0 & 0 & -3\end{pmatrix}$ \\
\hline
\end{tabular}
\end{center}
\end{solucio}


\begin{enunciat}
Es considera el pla vectorial $P$ de $\rn 3$
\[
  P = \langle (1,-2,-1), (2,-1,-1) \rangle
\]
i la base $\cal B'=\{(1,0,-1),(2,2,-1),(-1,-1,1)\}$. Determineu l'equació implícita del pla $P$ respecte de la base $\cal B'$.
\end{enunciat}

\begin{quadricula}
\begin{tabular}{|C{5cm}{0.5cm}|}
\hline
 \\
\hline
\end{tabular}
\end{quadricula}

\begin{solucio}
\begin{center}
\begin{tabular}{|C{5cm}{0.5cm}|}
\hline
$- 2 x' - 3 y' + 3 z' = 0$ \\
\hline
\end{tabular}
\end{center}
\end{solucio}


\begin{enunciat}
Resoleu el següent sistema d'equacions
\[
  \left.\aligned - 2 x + 2 y + z &= 2 \\ - x + y - z &= -5 \\ x + z &= 3 \\ 2 x - y + 2 z &= 8 \endaligned\;\right\}\,,
\]
\end{enunciat}

\begin{quadricula}
\begin{tabular}{|C{5cm}{0.5cm}|}
\hline
  \\
\hline
\end{tabular}
\end{quadricula}

\begin{solucio}
\begin{center}
\begin{tabular}{|C{5cm}{0.5cm}|}
\hline
$x=-1$, $y=-2$, $z=4$. \\
\hline
\end{tabular}
\end{center}
\end{solucio}


\begin{enunciat}
Determineu l'equació implícita del pla perpendicular a la recta 
\[
  x - 1 = \frac{y - 1}{3} = \frac{z + 4}{-4}
\]
que passa pel punt $(5,3,-2)$.
\end{enunciat}

\begin{quadricula}
\begin{tabular}{|C{5cm}{0.5cm}|}
\hline
  \\
\hline
\end{tabular}
\end{quadricula}

\begin{solucio}
\begin{center}
\begin{tabular}{|C{5cm}{0.5cm}|}
\hline
$x + 3 y - 4 z = 22$ \\
\hline
\end{tabular}
\end{center}
\end{solucio}


\end{document}