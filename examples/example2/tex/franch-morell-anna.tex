\documentclass[10pt,catalan]{article}

\usepackage[ca,NIC]{examen}

\data{15 de gener de 2021}
\examen{Examen Final}
\quadrimestre{Tardor}
\curs{2020-2021}
\assignatura{Àlgebra}
\cognoms{Franch Morell}
\nom{Anna}

\begin{asydef}
import eixos;
import graph;
usepackage("amsfonts");
usepackage("amssymb");
usepackage("times");
usepackage("txfonts");
\end{asydef}

\begin{document}

\begin{enunciat}
Donada la matriu
\[
   A=\begin{pmatrix}{*{3}r} 1 & 2 & 2\\ -2 & -3 & -2\\ -2 & -1 & -4\end{pmatrix}\,,
\]
\begin{apartats}
\item calculeu el seu poliniomi característic;
\item justifiqueu que la matriu $A$ és diagonalitzable i escriviu la matriu diagonal corresponent  ordenants els valors propis de menor a major.
\end{apartats}
\end{enunciat}

\begin{quadricula}
\begin{tabular}{|L{5cm}{2cm}|L{9cm}{2cm}|}
\hline
\multicolumn{2}{|L{14cm}{0.8cm}|}{$p(\lambda)=$} \\
\hline
$D=$ & \begin{minipage}[t]{8.8cm}Justificació:\vspace{1.8cm} \end{minipage}\\
\hline
\end{tabular}
\end{quadricula}

\begin{solucio}
\begin{center}
\begin{tabular}{|L{5cm}{2cm}|L{9cm}{2cm}|}
\hline
\multicolumn{2}{|L{14cm}{0.8cm}|}{$p(\lambda)=- \lambda^{3} - 6 \lambda^{2} - 11 \lambda - 6$} \\
\hline
$D=\begin{pmatrix}{*{3}r} -3 & 0 & 0\\ 0 & -2 & 0\\ 0 & 0 & -1\end{pmatrix}$ & 
\begin{minipage}[t]{8.8cm}
Justificació: tota matriu amb valors propis diferents o de multiplicitat 1, és diagonalitzable.
\vspace{1cm}
\end{minipage} \\
\hline
\end{tabular}
\end{center}
\end{solucio}





\begin{enunciat}
Donat el sistema d'equacions
\[
  \left.\aligned - 3 x + 4 y - 4 z &= m - 17 \\ 2 x - 2 y + m z &= 8 \\ x + 4 y + 4 z &= -6 \endaligned\;\right\}\,,
\]
\begin{apartats}
\item trobeu el valor o valors de $m$ per al qual el sistema és compatible indeterminat;
\item resoleu-lo per a aquests valors de $m$.
\end{apartats}
\end{enunciat}

\begin{quadricula}
\begin{tabular}{|L{3cm}{1.5cm}|L{7cm}{1.5cm}|}
\hline
$m=$ & \\
\hline
\end{tabular}
\end{quadricula}

\begin{solucio}
\begin{center}
\begin{tabular}{|L{3cm}{1.5cm}|L{7cm}{1.5cm}|}
\hline
$m=3$ & $\left.\aligned x &= - 2 z + 2 \\ y &= \frac{- z -4}{2} \endaligned\;\right\}$\\
\hline
\end{tabular}
\end{center}
\end{solucio}





\begin{enunciat}
Siguin $R$ la recta d'equació contínua
\[
  x - 1 = y - 2 = \frac{z - 2}{-1}\,.
\]
\begin{apartats}
\item Quina és la representació en la referència canònica del moviment helicoidal que consisteix en una rotacio d'angle $120^\circ$ al voltant de la recta $R$ seguida d'una translació de $(3,3,-3)$.
\item Trobeu els angles d'Euler de la rotació d'angle $120^\circ$ al voltant del vector $(1,1,-1)$.
\end{apartats}
\end{enunciat}

\begin{quadricula}
\begin{tabular}{|L{10cm}{2cm}|}
\hline
  \\
\hline
\multicolumn{1}{|C{10cm}{0.8cm}|}
{$\psi=$\hspace{3.5cm}$\theta=$\hspace{3.5cm}$\phi=$\hspace{3.5cm}} \\
\hline
\end{tabular}
\end{quadricula}

\begin{solucio}
\begin{center}
\begin{tabular}{|C{10cm}{2cm}|}
\hline
$\begin{pmatrix}{c} u \\ v \\ w\end{pmatrix} = 
\begin{pmatrix}{*{1}r} 2\\ 7\\ 0\end{pmatrix} + 
\begin{pmatrix}{*{3}r} 0 & 1 & 0\\ 0 & 0 & -1\\ -1 & 0 & 0\end{pmatrix}
\begin{pmatrix}{c} x \\ y \\ z\end{pmatrix}$ \\
\hline
\multicolumn{1}{|C{10cm}{0.8cm}|}
{$\psi=-90^\circ$\hspace{1.7cm}$\theta=90^\circ$\hspace{1.7cm}$\phi=0^\circ$} \\
\hline
\end{tabular}
\end{center}
\end{solucio}





\begin{enunciat}
Siguin $R_1$ i $R_2$ les rectes d'equacions respectives
\[
  \frac{x + 3}{3} = \frac{y + 2}{-1} = \frac{z + 2}{3}\qquad\text{i}\qquad \left.\aligned - 4 x + y - z &= 2 \\ 10 x - 2 y + 3 z &= -5 \endaligned\;\right\}\,.
\]
\begin{apartats}
\item Calculeu la distància entre les dues rectes.
\item Trobeu l'equació del pla que conté la primera recta i és paral·lel a la segona. 
\end{apartats}
\end{enunciat}

\begin{quadricula}
\begin{tabular}{|L{5cm}{0.8cm}|L{7cm}{0.8cm}|}
\hline
$d=$ & \\ 
\hline
\end{tabular}
\end{quadricula}

\begin{solucio}
\begin{center}
\begin{tabular}{|L{5cm}{0.8cm}|C{7cm}{0.8cm}|}
\hline
$d=\displaystyle \frac{11 \sqrt{146}}{73}$ & $4 x - 9 y - 7 z = 20$ \\
\hline
\end{tabular}
\end{center}
\end{solucio}





\begin{enunciat}
Donada la cònica d'equació
\[
  2 x^{2} + 2 y^{2} - 6 x y - 14 x + 16 y + 2 = 0
\]
\begin{apartats}
\item trobeu la seva referència principal, l'equació reduïda i el tipus de cònica;
\item representeu-la gràficament.
\end{apartats}
\end{enunciat}

\begin{quadricula}
\begin{tabular}{|L{16cm}{1cm}|}
\hline
  \\
\hline
\end{tabular}
\begin{figure}[!t]
\begin{center}
\begin{asy}
unitsize(0.45cm);

Canonica(-13,13,-10,10);

\end{asy}
\end{center}
\caption{Representació gràfica de la cònica}
\end{figure}
\end{quadricula}

\begin{solucio}
\begin{figure}[!t]
\begin{center}
\begin{asy}
unitsize(0.45cm);

Canonica(-11,15,-11,9);
Hiperbola((2,-1),(1,-1),4,20,x=10,y=10);
clip((-11,-11)--(15,-11)--(15,9)--(-11,9)--cycle);
\end{asy}
\end{center}
\caption{Representació gràfica de la cònica}
\end{figure}
\begin{center}
\begin{tabular}{|C{16cm}{1cm}|}
\hline
$\cal R'=\left\{(2,-1);\deufrac{1}{\sqrt{2}}(1,-1),\deufrac{1}{\sqrt{2}}(1,1)\right\}$\hspace{2cm} $\displaystyle \frac{x'^2}{4} - \frac{y'^2}{20} = 1$  \hspace{2cm} Hipèrbola\\
\hline
\end{tabular}
\end{center}
\end{solucio}






\end{document}
