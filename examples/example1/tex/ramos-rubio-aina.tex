\documentclass[11pt,catalan]{article}

\usepackage[ca,NIC]{examen}

\data{11 de novembre de 2020}
\examen{Primer Parcial}
\quadrimestre{Tardor}
\curs{2020-2021}
\assignatura{\`Algebra Lineal}
\cognoms{Ramos Rubio}
\nom{Aina}

\begin{document}

\begin{enunciat}
Trobeu la inversa de la matriu 
\[
  A = \begin{pmatrix}{*{3}r} -1 & 0 & 1\\ -1 & 2 & 0\\ 0 & 0 & 1\end{pmatrix}\,.
\]
\end{enunciat}

\begin{quadricula}
\begin{tabular}{|L{5cm}{1.25cm}|}
\hline
$A^{-1}=$ \\
\hline
\end{tabular}
\end{quadricula}

\begin{solucio}
\begin{center}
\begin{tabular}{|C{5cm}{1.25cm}|}
\hline
$A^{-1}=\deufrac{1}{2}\begin{pmatrix}{*{3}r} -2 & 0 & 2\\ -1 & 1 & 1\\ 0 & 0 & 2\end{pmatrix}$ \\
\hline
\end{tabular}
\end{center}
\end{solucio}


\begin{enunciat}
Es considera el pla vectorial $P$ de $\rn 3$
\[
  P = \langle (-1,1,1), (2,2,1) \rangle
\]
i la base $\cal B'=\{(-2,0,-1),(2,-1,2),(-1,1,-1)\}$. Determineu l'equació implícita del pla $P$ respecte de la base $\cal B'$.
\end{enunciat}

\begin{quadricula}
\begin{tabular}{|C{5cm}{0.5cm}|}
\hline
 \\
\hline
\end{tabular}
\end{quadricula}

\begin{solucio}
\begin{center}
\begin{tabular}{|C{5cm}{0.5cm}|}
\hline
$6 x' - 13 y' + 8 z' = 0$ \\
\hline
\end{tabular}
\end{center}
\end{solucio}


\begin{enunciat}
Resoleu el següent sistema d'equacions
\[
  \left.\aligned 2 x - y + z &= -1 \\ 2 x - 3 y + z &= 7 \\ - 2 x + 2 y - z &= -3 \\ x - y + z &= 3 \endaligned\;\right\}\,,
\]
\end{enunciat}

\begin{quadricula}
\begin{tabular}{|C{5cm}{0.5cm}|}
\hline
  \\
\hline
\end{tabular}
\end{quadricula}

\begin{solucio}
\begin{center}
\begin{tabular}{|C{5cm}{0.5cm}|}
\hline
$x=-4$, $y=-4$, $z=3$. \\
\hline
\end{tabular}
\end{center}
\end{solucio}


\begin{enunciat}
Determineu l'equació implícita del pla perpendicular a la recta 
\[
  x - 3 = \frac{y - 4}{3} = z - 4
\]
que passa pel punt $(1,-2,-2)$.
\end{enunciat}

\begin{quadricula}
\begin{tabular}{|C{5cm}{0.5cm}|}
\hline
  \\
\hline
\end{tabular}
\end{quadricula}

\begin{solucio}
\begin{center}
\begin{tabular}{|C{5cm}{0.5cm}|}
\hline
$x + 3 y + z = -7$ \\
\hline
\end{tabular}
\end{center}
\end{solucio}


\end{document}