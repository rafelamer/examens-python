\documentclass[11pt,catalan]{article}

\usepackage[ca,NIC]{examen}

\data{11 de novembre de 2020}
\examen{Primer Parcial}
\quadrimestre{Tardor}
\curs{2020-2021}
\assignatura{\`Algebra Lineal}
\cognoms{Sans Ramírez}
\nom{Gemma}

\begin{document}

\begin{enunciat}
Trobeu la inversa de la matriu 
\[
  A = \begin{pmatrix}{*{3}r} -3 & 0 & -4\\ -3 & -1 & -4\\ 1 & 0 & 2\end{pmatrix}\,.
\]
\end{enunciat}

\begin{quadricula}
\begin{tabular}{|L{5cm}{1.25cm}|}
\hline
$A^{-1}=$ \\
\hline
\end{tabular}
\end{quadricula}

\begin{solucio}
\begin{center}
\begin{tabular}{|C{5cm}{1.25cm}|}
\hline
$A^{-1}=\deufrac{1}{2}\begin{pmatrix}{*{3}r} -2 & 0 & -4\\ 2 & -2 & 0\\ 1 & 0 & 3\end{pmatrix}$ \\
\hline
\end{tabular}
\end{center}
\end{solucio}


\begin{enunciat}
Es considera el pla vectorial $P$ de $\rn 3$
\[
  P = \langle (-1,1,2), (-1,2,2) \rangle
\]
i la base $\cal B'=\{(-1,2,-1),(2,-1,1),(2,0,1)\}$. Determineu l'equació implícita del pla $P$ respecte de la base $\cal B'$.
\end{enunciat}

\begin{quadricula}
\begin{tabular}{|C{5cm}{0.5cm}|}
\hline
 \\
\hline
\end{tabular}
\end{quadricula}

\begin{solucio}
\begin{center}
\begin{tabular}{|C{5cm}{0.5cm}|}
\hline
$- 3 x' + 5 y' + 5 z' = 0$ \\
\hline
\end{tabular}
\end{center}
\end{solucio}


\begin{enunciat}
Resoleu el següent sistema d'equacions
\[
  \left.\aligned 3 x - 3 y - 2 z &= -11 \\ x + 2 y &= 4 \\ 2 x - y - 2 z &= -3 \\ - 4 x + 5 y + 2 z &= 19 \endaligned\;\right\}\,,
\]
\end{enunciat}

\begin{quadricula}
\begin{tabular}{|C{5cm}{0.5cm}|}
\hline
  \\
\hline
\end{tabular}
\end{quadricula}

\begin{solucio}
\begin{center}
\begin{tabular}{|C{5cm}{0.5cm}|}
\hline
$x=-2$, $y=3$, $z=-2$. \\
\hline
\end{tabular}
\end{center}
\end{solucio}


\begin{enunciat}
Determineu l'equació implícita del pla perpendicular a la recta 
\[
  \frac{x - 5}{4} = \frac{y - 1}{-3} = \frac{z + 5}{-4}
\]
que passa pel punt $(2,1,3)$.
\end{enunciat}

\begin{quadricula}
\begin{tabular}{|C{5cm}{0.5cm}|}
\hline
  \\
\hline
\end{tabular}
\end{quadricula}

\begin{solucio}
\begin{center}
\begin{tabular}{|C{5cm}{0.5cm}|}
\hline
$- 4 x + 3 y + 4 z = 7$ \\
\hline
\end{tabular}
\end{center}
\end{solucio}


\end{document}