\documentclass[11pt,catalan]{article}

\usepackage[ca,nicsol]{examen}

\data{11 de novembre de 2020}
\examen{Primer Parcial}
\quadrimestre{Tardor}
\curs{2020-2021}
\assignatura{\`Algebra Lineal}
\cognoms{Hernández Molina}
\nom{Francesc}

\begin{document}

\begin{enunciat}
Trobeu la inversa de la matriu 
\[
  A = \begin{pmatrix}{*{3}r} 3 & 0 & -1\\ 0 & 1 & 1\\ 1 & 0 & -1\end{pmatrix}\,.
\]
\end{enunciat}

\begin{quadricula}
\begin{tabular}{|L{5cm}{1.25cm}|}
\hline
$A^{-1}=$ \\
\hline
\end{tabular}
\end{quadricula}

\begin{solucio}
\begin{center}
\begin{tabular}{|C{5cm}{1.25cm}|}
\hline
$A^{-1}=\deufrac{1}{2}\begin{pmatrix}{*{3}r} 1 & 0 & -1\\ -1 & 2 & 3\\ 1 & 0 & -3\end{pmatrix}$ \\
\hline
\end{tabular}
\end{center}
\end{solucio}


\begin{enunciat}
Es considera el pla vectorial $P$ de $\rn 3$
\[
  P = \langle (-1,-2,2), (1,2,1) \rangle
\]
i la base $\cal B'=\{(2,0,1),(1,-2,1),(1,-1,1)\}$. Determineu l'equació implícita del pla $P$ respecte de la base $\cal B'$.
\end{enunciat}

\begin{quadricula}
\begin{tabular}{|C{5cm}{0.5cm}|}
\hline
 \\
\hline
\end{tabular}
\end{quadricula}

\begin{solucio}
\begin{center}
\begin{tabular}{|C{5cm}{0.5cm}|}
\hline
$4 x' + 4 y' + 3 z' = 0$ \\
\hline
\end{tabular}
\end{center}
\end{solucio}


\begin{enunciat}
Resoleu el següent sistema d'equacions
\[
  \left.\aligned 2 y + z &= -2 \\ - x + y - z &= -3 \\ x + 2 y + z &= -3 \\ 2 x + y + 2 z &= 0 \endaligned\;\right\}\,,
\]
\end{enunciat}

\begin{quadricula}
\begin{tabular}{|C{5cm}{0.5cm}|}
\hline
  \\
\hline
\end{tabular}
\end{quadricula}

\begin{solucio}
\begin{center}
\begin{tabular}{|C{5cm}{0.5cm}|}
\hline
$x=-1$, $y=-2$, $z=2$. \\
\hline
\end{tabular}
\end{center}
\end{solucio}


\begin{enunciat}
Determineu l'equació implícita del pla perpendicular a la recta 
\[
  \frac{x - 5}{2} = \frac{y + 1}{-1} = \frac{z + 1}{-2}
\]
que passa pel punt $(4,4,5)$.
\end{enunciat}

\begin{quadricula}
\begin{tabular}{|C{5cm}{0.5cm}|}
\hline
  \\
\hline
\end{tabular}
\end{quadricula}

\begin{solucio}
\begin{center}
\begin{tabular}{|C{5cm}{0.5cm}|}
\hline
$- 2 x + y + 2 z = 6$ \\
\hline
\end{tabular}
\end{center}
\end{solucio}


\end{document}